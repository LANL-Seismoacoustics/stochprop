%% Generated by Sphinx.
\def\sphinxdocclass{report}
\documentclass[letterpaper,10pt,english]{sphinxmanual}
\ifdefined\pdfpxdimen
   \let\sphinxpxdimen\pdfpxdimen\else\newdimen\sphinxpxdimen
\fi \sphinxpxdimen=.75bp\relax

\PassOptionsToPackage{warn}{textcomp}
\usepackage[utf8]{inputenc}
\ifdefined\DeclareUnicodeCharacter
% support both utf8 and utf8x syntaxes
  \ifdefined\DeclareUnicodeCharacterAsOptional
    \def\sphinxDUC#1{\DeclareUnicodeCharacter{"#1}}
  \else
    \let\sphinxDUC\DeclareUnicodeCharacter
  \fi
  \sphinxDUC{00A0}{\nobreakspace}
  \sphinxDUC{2500}{\sphinxunichar{2500}}
  \sphinxDUC{2502}{\sphinxunichar{2502}}
  \sphinxDUC{2514}{\sphinxunichar{2514}}
  \sphinxDUC{251C}{\sphinxunichar{251C}}
  \sphinxDUC{2572}{\textbackslash}
\fi
\usepackage{cmap}
\usepackage[T1]{fontenc}
\usepackage{amsmath,amssymb,amstext}
\usepackage{babel}



\usepackage{times}
\expandafter\ifx\csname T@LGR\endcsname\relax
\else
% LGR was declared as font encoding
  \substitutefont{LGR}{\rmdefault}{cmr}
  \substitutefont{LGR}{\sfdefault}{cmss}
  \substitutefont{LGR}{\ttdefault}{cmtt}
\fi
\expandafter\ifx\csname T@X2\endcsname\relax
  \expandafter\ifx\csname T@T2A\endcsname\relax
  \else
  % T2A was declared as font encoding
    \substitutefont{T2A}{\rmdefault}{cmr}
    \substitutefont{T2A}{\sfdefault}{cmss}
    \substitutefont{T2A}{\ttdefault}{cmtt}
  \fi
\else
% X2 was declared as font encoding
  \substitutefont{X2}{\rmdefault}{cmr}
  \substitutefont{X2}{\sfdefault}{cmss}
  \substitutefont{X2}{\ttdefault}{cmtt}
\fi


\usepackage[Bjarne]{fncychap}
\usepackage{sphinx}

\fvset{fontsize=\small}
\usepackage{geometry}


% Include hyperref last.
\usepackage{hyperref}
% Fix anchor placement for figures with captions.
\usepackage{hypcap}% it must be loaded after hyperref.
% Set up styles of URL: it should be placed after hyperref.
\urlstyle{same}

\usepackage{sphinxmessages}
\setcounter{tocdepth}{2}



\title{stochprop Documentation}
\date{Jul 29, 2020}
\release{1.0}
\author{P. Blom}
\newcommand{\sphinxlogo}{\vbox{}}
\renewcommand{\releasename}{Release}
\makeindex
\begin{document}

\pagestyle{empty}
\sphinxmaketitle
\pagestyle{plain}
\sphinxtableofcontents
\pagestyle{normal}
\phantomsection\label{\detokenize{index::doc}}



\chapter{Contents}
\label{\detokenize{index:module-stochprop}}\label{\detokenize{index:contents}}\index{stochprop (module)@\spxentry{stochprop}\spxextra{module}}

\section{Authorship \& License Info}
\label{\detokenize{authorship:authorship-license-info}}\label{\detokenize{authorship:authorship}}\label{\detokenize{authorship::doc}}
stochprop is being developed and maintained by Dr. Philip Blom (pblom at lanl.gov)

License info here…


\section{Installation}
\label{\detokenize{installation:installation}}\label{\detokenize{installation:id1}}\label{\detokenize{installation::doc}}

\subsection{Anaconda}
\label{\detokenize{installation:anaconda}}
The installation of stochprop is ideally completed using pip through Anaconda to resolve and download the correct python libraries. If you don’t currently have anaconda installed
on your system, please do that first.  Anaconda can be downloaded from \sphinxurl{https://www.anaconda.com/distribution/}.


\subsection{Installing Dependencies}
\label{\detokenize{installation:installing-dependencies}}

\subsubsection{Propagation Modeling Methods}
\label{\detokenize{installation:propagation-modeling-methods}}
A subset of the stochprop methods require access to the  LANL InfraGA/GeoAc ray tracing methods as well as the NCPAprop normal mode methods.  Many of the
empirical orthogonal function (EOF) based atmospheric statistics methods can be used without these propagation tools, but full usage of stochprop requires them.
\begin{itemize}
\item {} 
InfraGA/GeoAc: \sphinxurl{https://github.com/LANL-Seismoacoustics/infraGA}

\item {} 
NCPAprop: \sphinxurl{https://github.com/chetzer-ncpa/ncpaprop}

\end{itemize}


\subsubsection{InfraPy Signal Analysis Methods}
\label{\detokenize{installation:infrapy-signal-analysis-methods}}
The propagation models constructed in stochprop are intended for use in the Bayesian Infrasonic Source Localization (BISL) and Spectral Yield Estimation (SpYE)
methods in the LANL InfraPy signal analysis software suite.  As with the InfraGA/GeoAc and NCPAprop linkages, many of the EOF\sphinxhyphen{}based atmospheric statistics methods
can be utilized without InfraPy, but full usage will require installation of InfraPy (\sphinxurl{https://github.com/LANL-Seismoacoustics/infrapy}).


\subsection{Installing stochprop}
\label{\detokenize{installation:installing-stochprop}}
Once Anaconda is installed, you can install stochprop using pip by navigating to the base directory of the package (there will be a file there
named setup.py).  Assuming InfraPy has been installed within a conda environment called infrapy\_env, it is recommended to install stochprop in the same environment using:

\begin{sphinxVerbatim}[commandchars=\\\{\}]
\PYGZgt{}\PYGZgt{} conda activate infrapy\PYGZus{}env
\PYGZgt{}\PYGZgt{} pip install \PYGZhy{}e .
\end{sphinxVerbatim}

Otherwise, a new conda environment should be created with the underlying dependencies and pip should be used to install there (work on this later):

\begin{sphinxVerbatim}[commandchars=\\\{\}]
\PYGZgt{}\PYGZgt{} conda env create \PYGZhy{}f stochprop\PYGZus{}env.yml
\end{sphinxVerbatim}

If this command executes correctly and finishes without errors, it should print out instructions on how to activate and deactivate the new environment:

To activate the environment, use:

\begin{sphinxVerbatim}[commandchars=\\\{\}]
\PYGZgt{}\PYGZgt{} conda activate stochprop\PYGZus{}env
\end{sphinxVerbatim}

To deactivate an active environment, use

\begin{sphinxVerbatim}[commandchars=\\\{\}]
\PYGZgt{}\PYGZgt{} conda deactivate
\end{sphinxVerbatim}


\subsection{Testing stochprop}
\label{\detokenize{installation:testing-stochprop}}
Once the installation is complete, you can test the methods by navigating to the /examples directory located in the base directory, and running:

\begin{sphinxVerbatim}[commandchars=\\\{\}]
\PYGZgt{}\PYGZgt{} python eof\PYGZus{}analysis.py
\PYGZgt{}\PYGZgt{} python atmo\PYGZus{}analysis.py
\end{sphinxVerbatim}

A set of propagation analyses are included, but require installation of infraGA/GeoAc and NCPAprop.  These analysis can be run to ensure linkages are
working between stochprop and the propagation libraries, but note that the simulation of propagation through even the example suite of atmosphere
takes a significant amount of time.


\section{Stochastic Propagation Analysis}
\label{\detokenize{analysis:stochastic-propagation-analysis}}\label{\detokenize{analysis:analysis}}\label{\detokenize{analysis::doc}}\begin{itemize}
\item {} 
Stochastic propagation models have been developed to improve network\sphinxhyphen{}level signal analysis for infrasonic signal including localization and yield estimation

\item {} 
The atmospheric state at a given time and location is uncertainty due to the dynamic and poorly sampled nature of the atmosphere

\item {} 
Propagation effects for infrasonic signals must also include this uncertainty in order to properly quantify uncertainty in analysis results

\item {} 
A methodology of constructing propagation statistics has been developed that identifies a suite of atmospheric states that characterize the possible space of scenarios, runs propagation simulations through each possible state, and builds statistical distributions for propagation effects

\end{itemize}

\begin{figure}[htbp]
\centering
\capstart

\noindent\sphinxincludegraphics[width=500\sphinxpxdimen]{{stochprop_fig1}.jpg}
\caption{Stochastic propagation models are constructing using a suite of possible atmospheric states, propagation modeling applied to each, and a statistical model describing the variability in the resulting set of predicted effects}\label{\detokenize{analysis:id1}}\end{figure}
\begin{itemize}
\item {} 
The tools included here provide a framework for constructing such models using other openly available software tools

\end{itemize}


\subsection{Empirical Orthogonal Function Analysis}
\label{\detokenize{analysis:empirical-orthogonal-function-analysis}}

\subsubsection{Empirical Orthogonal Function Analysis}
\label{\detokenize{analysis:eofs}}\begin{itemize}
\item {} 
Discussion of empirical orthogonal function expansions and use in quantifying atmospheric variability…

\end{itemize}


\subsection{Atmospheric Sampling, Fitting, and Perturbation}
\label{\detokenize{analysis:atmospheric-sampling-fitting-and-perturbation}}

\subsubsection{Atmospheric Sampling, Fitting, and Perturbation}
\label{\detokenize{analysis:sampling}}\begin{itemize}
\item {} 
Discussion of sampling, fitting, and perturbing atmospheric specifications…

\end{itemize}


\subsection{Propagation Statistics}
\label{\detokenize{analysis:propagation-statistics}}

\subsubsection{Propagation Statistics}
\label{\detokenize{analysis:propagation}}\begin{itemize}
\item {} 
Discussion of building propagation statistics for path geometry and transmission loss…

\end{itemize}


\subsection{Section Links}
\label{\detokenize{analysis:section-links}}\begin{quote}


\subsubsection{Empirical Orthogonal Function Analysis}
\label{\detokenize{eofs:empirical-orthogonal-function-analysis}}\label{\detokenize{eofs:eofs}}\label{\detokenize{eofs::doc}}\begin{itemize}
\item {} 
Atmospheric specifications are available through a number of repositories, but the most up to date is maintained by University of Mississippi’s National Center for Physical Acoustics (NCPA) at \sphinxurl{http://g2s.ncpa.olemiss.edu}

\item {} 
Pull atmospheric specifications…

\end{itemize}


\paragraph{Define Run Parameters}
\label{\detokenize{eofs:define-run-parameters}}\begin{itemize}
\item {} 
Discussion…

\end{itemize}

\begin{sphinxVerbatim}[commandchars=\\\{\}]
\PYG{n}{prof\PYGZus{}dir} \PYG{o}{=} \PYG{l+s+s2}{\PYGZdq{}}\PYG{l+s+s2}{dir/of/g2s/}\PYG{l+s+s2}{\PYGZdq{}}
\PYG{n}{prof\PYGZus{}prefix} \PYG{o}{=} \PYG{l+s+s2}{\PYGZdq{}}\PYG{l+s+s2}{g2stxt\PYGZus{}}\PYG{l+s+s2}{\PYGZdq{}}
\PYG{n}{year\PYGZus{}lims} \PYG{o}{=} \PYG{p}{[}\PYG{l+m+mi}{2010}\PYG{p}{,} \PYG{l+m+mi}{2016}\PYG{p}{]}
\PYG{n}{run\PYGZus{}id} \PYG{o}{=} \PYG{l+s+s2}{\PYGZdq{}}\PYG{l+s+s2}{example}\PYG{l+s+s2}{\PYGZdq{}}
\PYG{n}{eof\PYGZus{}cnt} \PYG{o}{=} \PYG{l+m+mi}{50}
\end{sphinxVerbatim}


\paragraph{Load Atmosphere Specifications and Building EOFs}
\label{\detokenize{eofs:load-atmosphere-specifications-and-building-eofs}}\begin{itemize}
\item {} 
Discussion

\end{itemize}


\paragraph{Compute Coefficients and Determine Seasonality}
\label{\detokenize{eofs:compute-coefficients-and-determine-seasonality}}\begin{itemize}
\item {} 
Discussion…

\end{itemize}


\paragraph{Generate Samples from a Coefficient Set}
\label{\detokenize{eofs:generate-samples-from-a-coefficient-set}}\begin{itemize}
\item {} 
Discussion…

\end{itemize}


\subsubsection{Atmospheric Sampling, Fitting, and Perturbation}
\label{\detokenize{sampling:atmospheric-sampling-fitting-and-perturbation}}\label{\detokenize{sampling:sampling}}\label{\detokenize{sampling::doc}}\begin{itemize}
\item {} 
Overview of building propagation statistics and their use in BISL and SpYE

\end{itemize}


\paragraph{Fitting an Atmospheric Specification using EOFs}
\label{\detokenize{sampling:fitting-an-atmospheric-specification-using-eofs}}\begin{itemize}
\item {} 
Stuff…

\item {} 
Test math input:

\end{itemize}
\begin{equation*}
\begin{split}y = m x + b\end{split}
\end{equation*}\begin{itemize}
\item {} 
Test inline math input, \(y = m x + b\), and then it stops?

\end{itemize}


\paragraph{Sampling Specifications using EOF Coefficient Distributions}
\label{\detokenize{sampling:sampling-specifications-using-eof-coefficient-distributions}}\begin{itemize}
\item {} 
Stuff…

\end{itemize}


\paragraph{Perturbing Specifications to Account for Uncertainty}
\label{\detokenize{sampling:perturbing-specifications-to-account-for-uncertainty}}\begin{itemize}
\item {} 
Stuff…

\end{itemize}


\subsubsection{Propagation Statistics}
\label{\detokenize{propagation:propagation-statistics}}\label{\detokenize{propagation:propagation}}\label{\detokenize{propagation::doc}}\begin{itemize}
\item {} 
Overview of building propagation statistics and their use in BISL and SpYE

\end{itemize}


\paragraph{Path Geometry Models (PGMs)}
\label{\detokenize{propagation:path-geometry-models-pgms}}\begin{itemize}
\item {} 
Stuff…

\end{itemize}


\paragraph{Transmission Loss Models (TLMs)}
\label{\detokenize{propagation:transmission-loss-models-tlms}}\begin{itemize}
\item {} 
Stuff…

\end{itemize}
\end{quote}


\section{API}
\label{\detokenize{stochprop:api}}\label{\detokenize{stochprop:id1}}\label{\detokenize{stochprop::doc}}

\subsection{Empirical Orthogonal Function Analysis}
\label{\detokenize{stochprop.eofs:empirical-orthogonal-function-analysis}}\label{\detokenize{stochprop.eofs::doc}}\phantomsection\label{\detokenize{stochprop.eofs:module-stochprop.eofs}}\index{stochprop.eofs (module)@\spxentry{stochprop.eofs}\spxextra{module}}\index{build\_atmo\_matrix() (in module stochprop.eofs)@\spxentry{build\_atmo\_matrix()}\spxextra{in module stochprop.eofs}}

\begin{fulllineitems}
\phantomsection\label{\detokenize{stochprop.eofs:stochprop.eofs.build_atmo_matrix}}\pysiglinewithargsret{\sphinxcode{\sphinxupquote{stochprop.eofs.}}\sphinxbfcode{\sphinxupquote{build\_atmo\_matrix}}}{\emph{path}, \emph{pattern=\textquotesingle{}*.met\textquotesingle{}}, \emph{skiprows=0}, \emph{ref\_alts=None}}{}
Read in a list of atmosphere files from the path location
matching a specified pattern for continued analysis.
\begin{quote}\begin{description}
\item[{Parameters}] \leavevmode\begin{description}
\item[{\sphinxstylestrong{path: string}}] \leavevmode
Path to the profiles to be loaded

\item[{\sphinxstylestrong{pattern: string}}] \leavevmode
Pattern defining the list of profiles in the path

\item[{\sphinxstylestrong{skiprows: int}}] \leavevmode
Number of header rows in the profiles

\item[{\sphinxstylestrong{ref\_alts: 1darray}}] \leavevmode
Reference altitudes if comparison is needed

\end{description}

\item[{Returns}] \leavevmode\begin{description}
\item[{A: 2darray}] \leavevmode
Atmosphere array of size M x (5 * N) for M atmospheres where each atmosphere samples N altitudes

\end{description}

\end{description}\end{quote}

\end{fulllineitems}

\index{build\_cdf() (in module stochprop.eofs)@\spxentry{build\_cdf()}\spxextra{in module stochprop.eofs}}

\begin{fulllineitems}
\phantomsection\label{\detokenize{stochprop.eofs:stochprop.eofs.build_cdf}}\pysiglinewithargsret{\sphinxcode{\sphinxupquote{stochprop.eofs.}}\sphinxbfcode{\sphinxupquote{build\_cdf}}}{\emph{pdf}, \emph{lims}, \emph{pnts=250}}{}
Compute the cumulative distribution of a pdf within specified limits
\begin{quote}\begin{description}
\item[{Parameters}] \leavevmode\begin{description}
\item[{\sphinxstylestrong{pdf: function}}] \leavevmode
Probability distribution function (PDF) for a single variable

\item[{\sphinxstylestrong{lims: 1darray}}] \leavevmode
Iterable containing lower and upper bound for integration

\item[{\sphinxstylestrong{pnts: int}}] \leavevmode
Number of points to consider in defining the cumulative distribution

\end{description}

\item[{Returns}] \leavevmode\begin{description}
\item[{cfd: interp1d}] \leavevmode
Interpolated results for the cdf

\end{description}

\end{description}\end{quote}

\end{fulllineitems}

\index{compute\_coeffs() (in module stochprop.eofs)@\spxentry{compute\_coeffs()}\spxextra{in module stochprop.eofs}}

\begin{fulllineitems}
\phantomsection\label{\detokenize{stochprop.eofs:stochprop.eofs.compute_coeffs}}\pysiglinewithargsret{\sphinxcode{\sphinxupquote{stochprop.eofs.}}\sphinxbfcode{\sphinxupquote{compute\_coeffs}}}{\emph{A}, \emph{alts}, \emph{eofs\_path}, \emph{output\_path}, \emph{eof\_cnt=100}, \emph{pool=None}}{}
Compute the EOF coefficients for a suite of atmospheres
and store the coefficient values.
\begin{quote}\begin{description}
\item[{Parameters}] \leavevmode\begin{description}
\item[{\sphinxstylestrong{A: 2darray}}] \leavevmode
Suite of atmosphere specifications from build\_atmo\_matrix

\item[{\sphinxstylestrong{alts: 1darray}}] \leavevmode
Altitudes at which the atmosphere is sampled from build\_atmo\_matrix

\item[{\sphinxstylestrong{eofs\_path: string}}] \leavevmode
Path to the .eof results from compute\_svd

\item[{\sphinxstylestrong{output\_path: string}}] \leavevmode
Path where output will be stored

\item[{\sphinxstylestrong{eof\_cnt: int}}] \leavevmode
Number of EOFs to consider in computing coefficients

\item[{\sphinxstylestrong{pool: pathos.multiprocessing.ProcessingPool}}] \leavevmode
Multiprocessing pool for accelerating calculations

\end{description}

\item[{Returns}] \leavevmode\begin{description}
\item[{coeffs: 2darray}] \leavevmode
Array containing coefficient values of size prof\_cnt by eof\_cnt.  Result is also written to file.

\end{description}

\end{description}\end{quote}

\end{fulllineitems}

\index{compute\_overlap() (in module stochprop.eofs)@\spxentry{compute\_overlap()}\spxextra{in module stochprop.eofs}}

\begin{fulllineitems}
\phantomsection\label{\detokenize{stochprop.eofs:stochprop.eofs.compute_overlap}}\pysiglinewithargsret{\sphinxcode{\sphinxupquote{stochprop.eofs.}}\sphinxbfcode{\sphinxupquote{compute\_overlap}}}{\emph{coeffs}, \emph{eof\_cnt=100}}{}
Compute the overlap of EOF coefficient distributions
\begin{quote}\begin{description}
\item[{Parameters}] \leavevmode\begin{description}
\item[{\sphinxstylestrong{coeffs: list of 2darrays}}] \leavevmode\begin{description}
\item[{List of 2darrays containing coefficients to consider}] \leavevmode
overlap in PDF of values

\end{description}

\item[{\sphinxstylestrong{eof\_cnt: int}}] \leavevmode
Number of EOFs to compute

\end{description}

\item[{Returns}] \leavevmode\begin{description}
\item[{overlap: 3darray}] \leavevmode
Array containing overlap values of size coeff\_cnt by coeff\_cnt by eof\_cnt

\end{description}

\end{description}\end{quote}

\end{fulllineitems}

\index{compute\_seasonality() (in module stochprop.eofs)@\spxentry{compute\_seasonality()}\spxextra{in module stochprop.eofs}}

\begin{fulllineitems}
\phantomsection\label{\detokenize{stochprop.eofs:stochprop.eofs.compute_seasonality}}\pysiglinewithargsret{\sphinxcode{\sphinxupquote{stochprop.eofs.}}\sphinxbfcode{\sphinxupquote{compute\_seasonality}}}{\emph{overlap\_file}, \emph{eofs\_path}, \emph{file\_id=None}}{}
Compute the overlap of EOF coefficients to identify seasonality
\begin{quote}\begin{description}
\item[{Parameters}] \leavevmode\begin{description}
\item[{\sphinxstylestrong{overlap\_file: string}}] \leavevmode
Path and name of file containing results of stochprop.eofs.compute\_overlap

\item[{\sphinxstylestrong{eofs\_path: string}}] \leavevmode
Path to the .eof results from compute\_svd

\item[{\sphinxstylestrong{file\_id: string}}] \leavevmode
Path and ID to save the dendrogram result of the overlap analysis

\end{description}

\end{description}\end{quote}

\end{fulllineitems}

\index{compute\_svd() (in module stochprop.eofs)@\spxentry{compute\_svd()}\spxextra{in module stochprop.eofs}}

\begin{fulllineitems}
\phantomsection\label{\detokenize{stochprop.eofs:stochprop.eofs.compute_svd}}\pysiglinewithargsret{\sphinxcode{\sphinxupquote{stochprop.eofs.}}\sphinxbfcode{\sphinxupquote{compute\_svd}}}{\emph{A}, \emph{alts}, \emph{output\_path}, \emph{eof\_cnt=100}}{}
Computes the singular value decomposition (SVD)
of an atmosphere set read into an array by
stochprop.eofs.build\_atmo\_matrix() and saves
the basis functions (empirical orthogonal
functions) and singular values to file
\begin{quote}\begin{description}
\item[{Parameters}] \leavevmode\begin{description}
\item[{\sphinxstylestrong{A: 2darray}}] \leavevmode
Suite of atmosphere specifications from build\_atmo\_matrix

\item[{\sphinxstylestrong{alts: 1darray}}] \leavevmode
Altitudes at which the atmosphere is sampled from build\_atmo\_matrix

\item[{\sphinxstylestrong{output\_path: string}}] \leavevmode
Path to output the SVD results

\item[{\sphinxstylestrong{eof\_cnt: int}}] \leavevmode
Number of basic functions to save

\end{description}

\end{description}\end{quote}

\end{fulllineitems}

\index{define\_coeff\_limits() (in module stochprop.eofs)@\spxentry{define\_coeff\_limits()}\spxextra{in module stochprop.eofs}}

\begin{fulllineitems}
\phantomsection\label{\detokenize{stochprop.eofs:stochprop.eofs.define_coeff_limits}}\pysiglinewithargsret{\sphinxcode{\sphinxupquote{stochprop.eofs.}}\sphinxbfcode{\sphinxupquote{define\_coeff\_limits}}}{\emph{coeff\_vals}}{}
Compute upper and lower bounds for coefficient values
\begin{quote}\begin{description}
\item[{Parameters}] \leavevmode\begin{description}
\item[{\sphinxstylestrong{coeff\_vals: 2darrays}}] \leavevmode
Coefficients computed with stochprop.eofs.compute\_coeffs

\end{description}

\item[{Returns}] \leavevmode\begin{description}
\item[{lims: 1darray}] \leavevmode
Lower and upper bounds of coefficient value distribution

\end{description}

\end{description}\end{quote}

\end{fulllineitems}

\index{draw\_from\_pdf() (in module stochprop.eofs)@\spxentry{draw\_from\_pdf()}\spxextra{in module stochprop.eofs}}

\begin{fulllineitems}
\phantomsection\label{\detokenize{stochprop.eofs:stochprop.eofs.draw_from_pdf}}\pysiglinewithargsret{\sphinxcode{\sphinxupquote{stochprop.eofs.}}\sphinxbfcode{\sphinxupquote{draw\_from\_pdf}}}{\emph{pdf}, \emph{lims}, \emph{cdf=None}, \emph{size=1}}{}
Sample a number of values from a probability distribution
function (pdf) with specified limits
\begin{quote}\begin{description}
\item[{Parameters}] \leavevmode\begin{description}
\item[{\sphinxstylestrong{pdf: function}}] \leavevmode
Probability distribution function (PDF) for a single variable

\item[{\sphinxstylestrong{lims: 1darray}}] \leavevmode
Iterable containing lower and upper bound for integration

\item[{\sphinxstylestrong{cdf: function}}] \leavevmode
Cumulative distribution function (CDF) from stochprop.eofs.build\_cfd

\item[{\sphinxstylestrong{size: int}}] \leavevmode
Number of samples to generate

\end{description}

\item[{Returns}] \leavevmode\begin{description}
\item[{samples: 1darray}] \leavevmode
Sampled values from the PDF

\end{description}

\end{description}\end{quote}

\end{fulllineitems}

\index{fit\_atmo() (in module stochprop.eofs)@\spxentry{fit\_atmo()}\spxextra{in module stochprop.eofs}}

\begin{fulllineitems}
\phantomsection\label{\detokenize{stochprop.eofs:stochprop.eofs.fit_atmo}}\pysiglinewithargsret{\sphinxcode{\sphinxupquote{stochprop.eofs.}}\sphinxbfcode{\sphinxupquote{fit\_atmo}}}{\emph{prof\_path}, \emph{eofs\_path}, \emph{output\_path}, \emph{eof\_cnt=100}}{}
Compute a given number of EOF coefficients to fit a given
atmophere specification using the basic functions.  Write
the resulting approximated atmospheric specification to
file.
\begin{quote}\begin{description}
\item[{Parameters}] \leavevmode\begin{description}
\item[{\sphinxstylestrong{prof\_path: string}}] \leavevmode
Path and name of the specification to be fit

\item[{\sphinxstylestrong{eofs\_path: string}}] \leavevmode
Path to the .eof results from compute\_svd

\item[{\sphinxstylestrong{output\_path: string}}] \leavevmode
Path where output will be stored

\item[{\sphinxstylestrong{eof\_cnt: int}}] \leavevmode
Number of EOFs to use in building approximate specification

\end{description}

\end{description}\end{quote}

\end{fulllineitems}

\index{maximum\_likelihood\_profile() (in module stochprop.eofs)@\spxentry{maximum\_likelihood\_profile()}\spxextra{in module stochprop.eofs}}

\begin{fulllineitems}
\phantomsection\label{\detokenize{stochprop.eofs:stochprop.eofs.maximum_likelihood_profile}}\pysiglinewithargsret{\sphinxcode{\sphinxupquote{stochprop.eofs.}}\sphinxbfcode{\sphinxupquote{maximum\_likelihood\_profile}}}{\emph{coeffs}, \emph{eofs\_path}, \emph{output\_path}, \emph{eof\_cnt=100}}{}
Use coefficient distributions for a set of empirical orthogonal
basis functions to compute the maximum likelihood specification
\begin{quote}\begin{description}
\item[{Parameters}] \leavevmode\begin{description}
\item[{\sphinxstylestrong{coeffs: 2darrays}}] \leavevmode
Coefficients computed with stochprop.eofs.compute\_coeffs

\item[{\sphinxstylestrong{eofs\_path: string}}] \leavevmode
Path to the .eof results from compute\_svd

\item[{\sphinxstylestrong{output\_path: string}}] \leavevmode
Path where output will be stored

\item[{\sphinxstylestrong{eof\_cnt: int}}] \leavevmode
Number of EOFs to use in building sampled specifications

\end{description}

\end{description}\end{quote}

\end{fulllineitems}

\index{perturb\_atmo() (in module stochprop.eofs)@\spxentry{perturb\_atmo()}\spxextra{in module stochprop.eofs}}

\begin{fulllineitems}
\phantomsection\label{\detokenize{stochprop.eofs:stochprop.eofs.perturb_atmo}}\pysiglinewithargsret{\sphinxcode{\sphinxupquote{stochprop.eofs.}}\sphinxbfcode{\sphinxupquote{perturb\_atmo}}}{\emph{prof\_path}, \emph{eofs\_path}, \emph{output\_path}, \emph{uncertainty=10.0}, \emph{eof\_max=100}, \emph{eof\_cnt=50}, \emph{sample\_cnt=1}, \emph{alt\_wt\_pow=2.0}, \emph{sing\_val\_wt\_pow=0.25}}{}
Use EOFs to perturb a specified profile using a given scale
\begin{quote}\begin{description}
\item[{Parameters}] \leavevmode\begin{description}
\item[{\sphinxstylestrong{prof\_path: string}}] \leavevmode
Path and name of the specification to be fit

\item[{\sphinxstylestrong{eofs\_path: string}}] \leavevmode
Path to the .eof results from compute\_svd

\item[{\sphinxstylestrong{output\_path: string}}] \leavevmode
Path where output will be stored

\item[{\sphinxstylestrong{uncertainty: float}}] \leavevmode
Estimate of uncertainty in wind speeds; 95\% confidence is set to this value

\item[{\sphinxstylestrong{eof\_max: int}}] \leavevmode
Higher numbered EOF to sample

\item[{\sphinxstylestrong{eof\_cnt: int}}] \leavevmode
Number of EOFs to sample in the perturbation (can be less than eof\_max)

\item[{\sphinxstylestrong{sample\_cnt: int}}] \leavevmode
Number of perturbed atmospheric samples to generate

\item[{\sphinxstylestrong{alt\_wt\_pow: float}}] \leavevmode
Power raising relative mean altitude value in weighting

\item[{\sphinxstylestrong{sing\_val\_wt\_pow: float}}] \leavevmode
Power raising relative singular value in weighting

\end{description}

\end{description}\end{quote}

\end{fulllineitems}

\index{profiles\_qc() (in module stochprop.eofs)@\spxentry{profiles\_qc()}\spxextra{in module stochprop.eofs}}

\begin{fulllineitems}
\phantomsection\label{\detokenize{stochprop.eofs:stochprop.eofs.profiles_qc}}\pysiglinewithargsret{\sphinxcode{\sphinxupquote{stochprop.eofs.}}\sphinxbfcode{\sphinxupquote{profiles\_qc}}}{\emph{path}, \emph{pattern=\textquotesingle{}*.met\textquotesingle{}}, \emph{skiprows=0}}{}
Runs a quality control (QC) check on profiles in the path
matching the pattern.  It can optionally plot the bad
profiles.  If it finds any, it makes a new direcotry
in the path location called “bad\_profs” and moves those
profiles into the directory for you to check
\begin{quote}\begin{description}
\item[{Parameters}] \leavevmode\begin{description}
\item[{\sphinxstylestrong{path: string}}] \leavevmode
Path to the profiles to be QC’d

\item[{\sphinxstylestrong{pattern: string}}] \leavevmode
Pattern defining the list of profiles in the path

\item[{\sphinxstylestrong{skiprows: int}}] \leavevmode
Number of header rows in the profiles

\end{description}

\end{description}\end{quote}

\end{fulllineitems}

\index{sample\_atmo() (in module stochprop.eofs)@\spxentry{sample\_atmo()}\spxextra{in module stochprop.eofs}}

\begin{fulllineitems}
\phantomsection\label{\detokenize{stochprop.eofs:stochprop.eofs.sample_atmo}}\pysiglinewithargsret{\sphinxcode{\sphinxupquote{stochprop.eofs.}}\sphinxbfcode{\sphinxupquote{sample\_atmo}}}{\emph{coeffs}, \emph{eofs\_path}, \emph{output\_path}, \emph{eof\_cnt=100}, \emph{prof\_cnt=250}, \emph{output\_mean=False}}{}
Generate atmosphere states using coefficient distributions for
a set of empirical orthogonal basis functions
\begin{quote}\begin{description}
\item[{Parameters}] \leavevmode\begin{description}
\item[{\sphinxstylestrong{coeffs: 2darrays}}] \leavevmode
Coefficients computed with stochprop.eofs.compute\_coeffs

\item[{\sphinxstylestrong{eofs\_path: string}}] \leavevmode
Path to the .eof results from compute\_svd

\item[{\sphinxstylestrong{output\_path: string}}] \leavevmode
Path where output will be stored

\item[{\sphinxstylestrong{eof\_cnt: int}}] \leavevmode
Number of EOFs to use in building sampled specifications

\item[{\sphinxstylestrong{prof\_cnt: int}}] \leavevmode
Number of atmospheric specification samples to generate

\item[{\sphinxstylestrong{output\_mean: bool}}] \leavevmode
Flag to output the mean profile from the samples generated

\end{description}

\end{description}\end{quote}

\end{fulllineitems}



\subsection{Propagation Statistics}
\label{\detokenize{stochprop.propagation:module-stochprop.propagation}}\label{\detokenize{stochprop.propagation:propagation-statistics}}\label{\detokenize{stochprop.propagation::doc}}\index{stochprop.propagation (module)@\spxentry{stochprop.propagation}\spxextra{module}}\index{PathGeometryModel (class in stochprop.propagation)@\spxentry{PathGeometryModel}\spxextra{class in stochprop.propagation}}

\begin{fulllineitems}
\phantomsection\label{\detokenize{stochprop.propagation:stochprop.propagation.PathGeometryModel}}\pysigline{\sphinxbfcode{\sphinxupquote{class }}\sphinxcode{\sphinxupquote{stochprop.propagation.}}\sphinxbfcode{\sphinxupquote{PathGeometryModel}}}
Bases: \sphinxcode{\sphinxupquote{object}}

Propagation path geometry statistics computed using ray tracing
analysis on a suite of specifications includes celerity\sphinxhyphen{}range and
azimuth deviation/scatter statistics
\subsubsection*{Methods}


\begin{savenotes}\sphinxatlongtablestart\begin{longtable}[c]{\X{1}{2}\X{1}{2}}
\hline

\endfirsthead

\multicolumn{2}{c}%
{\makebox[0pt]{\sphinxtablecontinued{\tablename\ \thetable{} \textendash{} continued from previous page}}}\\
\hline

\endhead

\hline
\multicolumn{2}{r}{\makebox[0pt][r]{\sphinxtablecontinued{Continued on next page}}}\\
\endfoot

\endlastfoot

{\hyperref[\detokenize{stochprop.propagation:stochprop.propagation.PathGeometryModel.build}]{\sphinxcrossref{\sphinxcode{\sphinxupquote{build}}}}}(arrivals\_file, output\_file{[}, …{]})
&
Construct propagation statistics from a ray tracing arrival file (concatenated from multiple runs most likely) and output a path geometry model
\\
\hline
{\hyperref[\detokenize{stochprop.propagation:stochprop.propagation.PathGeometryModel.display}]{\sphinxcrossref{\sphinxcode{\sphinxupquote{display}}}}}({[}file\_id, subtitle{]})
&
Display the propagation geometry statistics
\\
\hline
{\hyperref[\detokenize{stochprop.propagation:stochprop.propagation.PathGeometryModel.eval_az_dev_mn}]{\sphinxcrossref{\sphinxcode{\sphinxupquote{eval\_az\_dev\_mn}}}}}(rng, az)
&
Evaluate the mean back azimuth deviation at a given range and propagation azimuth
\\
\hline
{\hyperref[\detokenize{stochprop.propagation:stochprop.propagation.PathGeometryModel.eval_az_dev_std}]{\sphinxcrossref{\sphinxcode{\sphinxupquote{eval\_az\_dev\_std}}}}}(rng, az)
&
Evaluate the standard deviation of the back azimuth at a given range and propagation azimuth
\\
\hline
{\hyperref[\detokenize{stochprop.propagation:stochprop.propagation.PathGeometryModel.eval_rcel_gmm}]{\sphinxcrossref{\sphinxcode{\sphinxupquote{eval\_rcel\_gmm}}}}}(rng, rcel, az)
&
Evaluate reciprocal celerity Gaussian Mixture Model (GMM) at specified range, reciprocal celerity, and azimuth
\\
\hline
{\hyperref[\detokenize{stochprop.propagation:stochprop.propagation.PathGeometryModel.load}]{\sphinxcrossref{\sphinxcode{\sphinxupquote{load}}}}}(model\_file{[}, smooth{]})
&
Load a path geometry model file for use
\\
\hline
\end{longtable}\sphinxatlongtableend\end{savenotes}
\index{build() (stochprop.propagation.PathGeometryModel method)@\spxentry{build()}\spxextra{stochprop.propagation.PathGeometryModel method}}

\begin{fulllineitems}
\phantomsection\label{\detokenize{stochprop.propagation:stochprop.propagation.PathGeometryModel.build}}\pysiglinewithargsret{\sphinxbfcode{\sphinxupquote{build}}}{\emph{arrivals\_file, output\_file, show\_fits=False, rng\_width=50.0, rng\_spacing=10.0, geom=\textquotesingle{}3d\textquotesingle{}, src\_loc={[}0.0, 0.0, 0.0{]}, min\_turning\_ht=0.0}}{}
Construct propagation statistics from a ray tracing arrival file (concatenated from
multiple runs most likely) and output a path geometry model
\begin{quote}\begin{description}
\item[{Parameters}] \leavevmode\begin{description}
\item[{\sphinxstylestrong{arrivals\_file: string}}] \leavevmode
Path to file containing infraGA/GeoAc arrival information

\item[{\sphinxstylestrong{output\_file: string}}] \leavevmode
Path to file where results will be saved

\item[{\sphinxstylestrong{show\_fits: boolean}}] \leavevmode
Option ot visualize model construction (for QC purposes)

\item[{\sphinxstylestrong{rng\_width: float}}] \leavevmode
Range bin width in kilometers

\item[{\sphinxstylestrong{rng\_spacing: float}}] \leavevmode
Spacing between range bins in kilometers

\item[{\sphinxstylestrong{geom: string}}] \leavevmode
Geometry used in infraGA/GeoAc simulation.  Options are “3d” and “sph”

\item[{\sphinxstylestrong{src\_loc: iterable}}] \leavevmode
{[}x, y, z{]} or {[}lat, lon, elev{]} location of the source used in infraGA/GeoAc simulations.  Note: ‘3d’ simulations assume source at origin.

\item[{\sphinxstylestrong{min\_turning\_ht: float}}] \leavevmode
Minimum turning height used to filter out boundary layer paths if not of interest

\end{description}

\end{description}\end{quote}

\end{fulllineitems}

\index{display() (stochprop.propagation.PathGeometryModel method)@\spxentry{display()}\spxextra{stochprop.propagation.PathGeometryModel method}}

\begin{fulllineitems}
\phantomsection\label{\detokenize{stochprop.propagation:stochprop.propagation.PathGeometryModel.display}}\pysiglinewithargsret{\sphinxbfcode{\sphinxupquote{display}}}{\emph{file\_id=None}, \emph{subtitle=None}}{}
Display the propagation geometry statistics
\begin{quote}\begin{description}
\item[{Parameters}] \leavevmode\begin{description}
\item[{\sphinxstylestrong{file\_id: string}}] \leavevmode
File prefix to save visualization

\item[{\sphinxstylestrong{subtitle: string}}] \leavevmode
Subtitle used in figures

\end{description}

\end{description}\end{quote}

\end{fulllineitems}

\index{eval\_az\_dev\_mn() (stochprop.propagation.PathGeometryModel method)@\spxentry{eval\_az\_dev\_mn()}\spxextra{stochprop.propagation.PathGeometryModel method}}

\begin{fulllineitems}
\phantomsection\label{\detokenize{stochprop.propagation:stochprop.propagation.PathGeometryModel.eval_az_dev_mn}}\pysiglinewithargsret{\sphinxbfcode{\sphinxupquote{eval\_az\_dev\_mn}}}{\emph{rng}, \emph{az}}{}
Evaluate the mean back azimuth deviation at a given range
and propagation azimuth
\begin{quote}\begin{description}
\item[{Parameters}] \leavevmode\begin{description}
\item[{\sphinxstylestrong{rng: float}}] \leavevmode
Range from source

\item[{\sphinxstylestrong{az: float}}] \leavevmode
Propagation azimuth (relative to North)

\end{description}

\item[{Returns}] \leavevmode\begin{description}
\item[{bias: float}] \leavevmode
Predicted bias in the arrival back azimuth at specified arrival range and azimuth

\end{description}

\end{description}\end{quote}

\end{fulllineitems}

\index{eval\_az\_dev\_std() (stochprop.propagation.PathGeometryModel method)@\spxentry{eval\_az\_dev\_std()}\spxextra{stochprop.propagation.PathGeometryModel method}}

\begin{fulllineitems}
\phantomsection\label{\detokenize{stochprop.propagation:stochprop.propagation.PathGeometryModel.eval_az_dev_std}}\pysiglinewithargsret{\sphinxbfcode{\sphinxupquote{eval\_az\_dev\_std}}}{\emph{rng}, \emph{az}}{}
Evaluate the standard deviation of the back azimuth at a given range
and propagation azimuth
\begin{quote}\begin{description}
\item[{Parameters}] \leavevmode\begin{description}
\item[{\sphinxstylestrong{rng: float}}] \leavevmode
Range from source

\item[{\sphinxstylestrong{az: float}}] \leavevmode
Propagation azimuth (relative to North)

\end{description}

\item[{Returns}] \leavevmode\begin{description}
\item[{stdev: float}] \leavevmode
Standard deviation of arrival back azimuths at specified range and azimuth

\end{description}

\end{description}\end{quote}

\end{fulllineitems}

\index{eval\_rcel\_gmm() (stochprop.propagation.PathGeometryModel method)@\spxentry{eval\_rcel\_gmm()}\spxextra{stochprop.propagation.PathGeometryModel method}}

\begin{fulllineitems}
\phantomsection\label{\detokenize{stochprop.propagation:stochprop.propagation.PathGeometryModel.eval_rcel_gmm}}\pysiglinewithargsret{\sphinxbfcode{\sphinxupquote{eval\_rcel\_gmm}}}{\emph{rng}, \emph{rcel}, \emph{az}}{}
Evaluate reciprocal celerity Gaussian Mixture Model (GMM)
at specified range, reciprocal celerity, and azimuth
\begin{quote}\begin{description}
\item[{Parameters}] \leavevmode\begin{description}
\item[{\sphinxstylestrong{rng: float}}] \leavevmode
Range from source

\item[{\sphinxstylestrong{rcel: float}}] \leavevmode
Reciprocal celerity (travel time divided by propagation range)

\item[{\sphinxstylestrong{az: float}}] \leavevmode
Propagation azimuth (relative to North)

\end{description}

\item[{Returns}] \leavevmode\begin{description}
\item[{pdf: float}] \leavevmode
Probability of observing an infrasonic arrival with specified celerity at specified range and azimuth

\end{description}

\end{description}\end{quote}

\end{fulllineitems}

\index{load() (stochprop.propagation.PathGeometryModel method)@\spxentry{load()}\spxextra{stochprop.propagation.PathGeometryModel method}}

\begin{fulllineitems}
\phantomsection\label{\detokenize{stochprop.propagation:stochprop.propagation.PathGeometryModel.load}}\pysiglinewithargsret{\sphinxbfcode{\sphinxupquote{load}}}{\emph{model\_file}, \emph{smooth=False}}{}
Load a path geometry model file for use
\begin{quote}\begin{description}
\item[{Parameters}] \leavevmode\begin{description}
\item[{\sphinxstylestrong{model\_file: string}}] \leavevmode
Path to PGM file constructed using stochprop.propagation.PathGeometryModel.build()

\item[{\sphinxstylestrong{smooth: boolean}}] \leavevmode
Option to use scipy.signal.savgol\_filter to smooth discrete GMM parameters along range

\end{description}

\end{description}\end{quote}

\end{fulllineitems}


\end{fulllineitems}

\index{TLossModel (class in stochprop.propagation)@\spxentry{TLossModel}\spxextra{class in stochprop.propagation}}

\begin{fulllineitems}
\phantomsection\label{\detokenize{stochprop.propagation:stochprop.propagation.TLossModel}}\pysigline{\sphinxbfcode{\sphinxupquote{class }}\sphinxcode{\sphinxupquote{stochprop.propagation.}}\sphinxbfcode{\sphinxupquote{TLossModel}}}
Bases: \sphinxcode{\sphinxupquote{object}}
\subsubsection*{Methods}


\begin{savenotes}\sphinxatlongtablestart\begin{longtable}[c]{\X{1}{2}\X{1}{2}}
\hline

\endfirsthead

\multicolumn{2}{c}%
{\makebox[0pt]{\sphinxtablecontinued{\tablename\ \thetable{} \textendash{} continued from previous page}}}\\
\hline

\endhead

\hline
\multicolumn{2}{r}{\makebox[0pt][r]{\sphinxtablecontinued{Continued on next page}}}\\
\endfoot

\endlastfoot

{\hyperref[\detokenize{stochprop.propagation:stochprop.propagation.TLossModel.build}]{\sphinxcrossref{\sphinxcode{\sphinxupquote{build}}}}}(tloss\_file, output\_file{[}, show\_fits, …{]})
&
Construct propagation statistics from a NCPAprop modess or pape file (concatenated from multiple runs most likely) and output a transmission loss model
\\
\hline
{\hyperref[\detokenize{stochprop.propagation:stochprop.propagation.TLossModel.display}]{\sphinxcrossref{\sphinxcode{\sphinxupquote{display}}}}}({[}file\_id, title{]})
&
Display the transmission loss statistics
\\
\hline
{\hyperref[\detokenize{stochprop.propagation:stochprop.propagation.TLossModel.eval}]{\sphinxcrossref{\sphinxcode{\sphinxupquote{eval}}}}}(rng, tloss, az)
&
Evaluate TLoss model at specified range, transmission loss, and azimuth
\\
\hline
{\hyperref[\detokenize{stochprop.propagation:stochprop.propagation.TLossModel.load}]{\sphinxcrossref{\sphinxcode{\sphinxupquote{load}}}}}(model\_file)
&
Load a transmission loss file for use
\\
\hline
\end{longtable}\sphinxatlongtableend\end{savenotes}
\index{build() (stochprop.propagation.TLossModel method)@\spxentry{build()}\spxextra{stochprop.propagation.TLossModel method}}

\begin{fulllineitems}
\phantomsection\label{\detokenize{stochprop.propagation:stochprop.propagation.TLossModel.build}}\pysiglinewithargsret{\sphinxbfcode{\sphinxupquote{build}}}{\emph{tloss\_file}, \emph{output\_file}, \emph{show\_fits=False}, \emph{use\_coh=False}}{}
Construct propagation statistics from a NCPAprop modess or pape file (concatenated from
multiple runs most likely) and output a transmission loss model
\begin{quote}\begin{description}
\item[{Parameters}] \leavevmode\begin{description}
\item[{\sphinxstylestrong{tloss\_file: string}}] \leavevmode
Path to file containing NCPAprop transmission loss information

\item[{\sphinxstylestrong{output\_file: string}}] \leavevmode
Path to file where results will be saved

\item[{\sphinxstylestrong{show\_fits: boolean}}] \leavevmode
Option ot visualize model construction (for QC purposes)

\item[{\sphinxstylestrong{use\_coh: boolean}}] \leavevmode
Option to use coherent transmission loss

\end{description}

\end{description}\end{quote}

\end{fulllineitems}

\index{display() (stochprop.propagation.TLossModel method)@\spxentry{display()}\spxextra{stochprop.propagation.TLossModel method}}

\begin{fulllineitems}
\phantomsection\label{\detokenize{stochprop.propagation:stochprop.propagation.TLossModel.display}}\pysiglinewithargsret{\sphinxbfcode{\sphinxupquote{display}}}{\emph{file\_id=None}, \emph{title=\textquotesingle{}Transmission Loss Statistics\textquotesingle{}}}{}
Display the transmission loss statistics
\begin{quote}\begin{description}
\item[{Parameters}] \leavevmode\begin{description}
\item[{\sphinxstylestrong{file\_id: string}}] \leavevmode
File prefix to save visualization

\item[{\sphinxstylestrong{subtitle: string}}] \leavevmode
Subtitle used in figures

\end{description}

\end{description}\end{quote}

\end{fulllineitems}

\index{eval() (stochprop.propagation.TLossModel method)@\spxentry{eval()}\spxextra{stochprop.propagation.TLossModel method}}

\begin{fulllineitems}
\phantomsection\label{\detokenize{stochprop.propagation:stochprop.propagation.TLossModel.eval}}\pysiglinewithargsret{\sphinxbfcode{\sphinxupquote{eval}}}{\emph{rng}, \emph{tloss}, \emph{az}}{}
Evaluate TLoss model at specified range, transmission loss, and azimuth
\begin{quote}\begin{description}
\item[{Parameters}] \leavevmode\begin{description}
\item[{\sphinxstylestrong{rng: float}}] \leavevmode
Range from source

\item[{\sphinxstylestrong{tloss: float}}] \leavevmode
Transmission loss

\item[{\sphinxstylestrong{az: float}}] \leavevmode
Propagation azimuth (relative to North)

\end{description}

\item[{Returns}] \leavevmode\begin{description}
\item[{pdf: float}] \leavevmode
Probability of observing an infrasonic arrival with specified transmission loss at specified range and azimuth

\end{description}

\end{description}\end{quote}

\end{fulllineitems}

\index{load() (stochprop.propagation.TLossModel method)@\spxentry{load()}\spxextra{stochprop.propagation.TLossModel method}}

\begin{fulllineitems}
\phantomsection\label{\detokenize{stochprop.propagation:stochprop.propagation.TLossModel.load}}\pysiglinewithargsret{\sphinxbfcode{\sphinxupquote{load}}}{\emph{model\_file}}{}
Load a transmission loss file for use
\begin{quote}\begin{description}
\item[{Parameters}] \leavevmode\begin{description}
\item[{\sphinxstylestrong{model\_file: string}}] \leavevmode
Path to TLoss file constructed using stochprop.propagation.TLossModel.build()

\end{description}

\end{description}\end{quote}

\end{fulllineitems}


\end{fulllineitems}

\index{find\_azimuth\_bin() (in module stochprop.propagation)@\spxentry{find\_azimuth\_bin()}\spxextra{in module stochprop.propagation}}

\begin{fulllineitems}
\phantomsection\label{\detokenize{stochprop.propagation:stochprop.propagation.find_azimuth_bin}}\pysiglinewithargsret{\sphinxcode{\sphinxupquote{stochprop.propagation.}}\sphinxbfcode{\sphinxupquote{find\_azimuth\_bin}}}{\emph{az}, \emph{bin\_cnt=16}}{}
Identify the azimuth bin index given some specified number of bins
\begin{quote}\begin{description}
\item[{Parameters}] \leavevmode\begin{description}
\item[{\sphinxstylestrong{az: float}}] \leavevmode
Azimuth in degrees

\item[{\sphinxstylestrong{bin\_cnt: int}}] \leavevmode
Number of bins used in analysis

\end{description}

\item[{Returns}] \leavevmode\begin{description}
\item[{index: int}] \leavevmode
Index of azimuth bin

\end{description}

\end{description}\end{quote}

\end{fulllineitems}

\index{run\_infraga() (in module stochprop.propagation)@\spxentry{run\_infraga()}\spxextra{in module stochprop.propagation}}

\begin{fulllineitems}
\phantomsection\label{\detokenize{stochprop.propagation:stochprop.propagation.run_infraga}}\pysiglinewithargsret{\sphinxcode{\sphinxupquote{stochprop.propagation.}}\sphinxbfcode{\sphinxupquote{run\_infraga}}}{\emph{profs\_path, results\_file, pattern=\textquotesingle{}*.met\textquotesingle{}, cpu\_cnt=None, geom=\textquotesingle{}3d\textquotesingle{}, bounces=25, inclinations={[}1.0, 60.0, 1.0{]}, azimuths={[}\sphinxhyphen{}180.0, 180.0, 3.0{]}, freq=0.1, z\_grnd=0.0, rng\_max=1000.0, src\_loc={[}0.0, 0.0, 0.0{]}, infraga\_path=\textquotesingle{}\textquotesingle{}}}{}
Run the infraga \sphinxhyphen{}prop algorithm to compute path geometry
statistics for BISL using a suite of specifications
and combining results into single file
\begin{quote}\begin{description}
\item[{Parameters}] \leavevmode\begin{description}
\item[{\sphinxstylestrong{profs\_path: string}}] \leavevmode
Path to atmospheric specification files

\item[{\sphinxstylestrong{results\_file: string}}] \leavevmode
Path and name of file where results will be written

\item[{\sphinxstylestrong{pattern: string}}] \leavevmode
Pattern identifying atmospheric specification within profs\_path location

\item[{\sphinxstylestrong{cpu\_cnt: int}}] \leavevmode
Number of threads to use in OpenMPI implementation.  None runs non\sphinxhyphen{}OpenMPI version of infraga

\item[{\sphinxstylestrong{geom: string}}] \leavevmode
Defines geometry of the infraga simulations (3d” or “sph”)

\item[{\sphinxstylestrong{bounces: int}}] \leavevmode
Maximum number of ground reflections to consider in ray tracing

\item[{\sphinxstylestrong{inclinations: iterable object}}] \leavevmode
Iterable of starting, ending, and step for ray launch inclination

\item[{\sphinxstylestrong{azimuths: iterable object}}] \leavevmode
Iterable of starting, ending, and step for ray launch azimuths

\item[{\sphinxstylestrong{freq: float}}] \leavevmode
Frequency to use for Sutherland Bass absorption calculation

\item[{\sphinxstylestrong{z\_grnd: float}}] \leavevmode
Elevation of the ground surface relative to sea level

\item[{\sphinxstylestrong{rng\_max: float}}] \leavevmode
Maximum propagation range for propagation paths

\item[{\sphinxstylestrong{src\_loc: iterable object}}] \leavevmode
The horizontal (latitude and longitude) and altitude of the source

\item[{\sphinxstylestrong{infraga\_path: string}}] \leavevmode
Location of infraGA executables

\end{description}

\end{description}\end{quote}

\end{fulllineitems}

\index{run\_modess() (in module stochprop.propagation)@\spxentry{run\_modess()}\spxextra{in module stochprop.propagation}}

\begin{fulllineitems}
\phantomsection\label{\detokenize{stochprop.propagation:stochprop.propagation.run_modess}}\pysiglinewithargsret{\sphinxcode{\sphinxupquote{stochprop.propagation.}}\sphinxbfcode{\sphinxupquote{run\_modess}}}{\emph{profs\_path, results\_path, pattern=\textquotesingle{}*.met\textquotesingle{}, cpu\_cnt=None, azimuths={[}\sphinxhyphen{}180.0, 180.0, 3.0{]}, freq=0.1, z\_grnd=0.0, rng\_max=1000.0, ncpaprop\_path=\textquotesingle{}\textquotesingle{}}}{}
Run the NCPAprop normal mode methods to compute transmission
loss values for a suite of atmospheric specifications at
a set of frequency values
\begin{quote}\begin{description}
\item[{Parameters}] \leavevmode\begin{description}
\item[{\sphinxstylestrong{profs\_path: string}}] \leavevmode
Path to atmospheric specification files

\item[{\sphinxstylestrong{results\_file: string}}] \leavevmode
Path and name of file where results will be written

\item[{\sphinxstylestrong{pattern: string}}] \leavevmode
Pattern identifying atmospheric specification within profs\_path location

\item[{\sphinxstylestrong{azimuths: iterable object}}] \leavevmode
Iterable of starting, ending, and step for propagation azimuths

\item[{\sphinxstylestrong{freq: float}}] \leavevmode
Frequency for simulation

\item[{\sphinxstylestrong{z\_grnd: float}}] \leavevmode
Elevation of the ground surface relative to sea level

\item[{\sphinxstylestrong{rng\_max: float}}] \leavevmode
Maximum propagation range for propagation paths

\end{description}

\end{description}\end{quote}

\end{fulllineitems}



\renewcommand{\indexname}{Python Module Index}
\begin{sphinxtheindex}
\let\bigletter\sphinxstyleindexlettergroup
\bigletter{s}
\item\relax\sphinxstyleindexentry{stochprop}\sphinxstyleindexpageref{index:\detokenize{module-stochprop}}
\item\relax\sphinxstyleindexentry{stochprop.eofs}\sphinxstyleindexpageref{stochprop.eofs:\detokenize{module-stochprop.eofs}}
\item\relax\sphinxstyleindexentry{stochprop.propagation}\sphinxstyleindexpageref{stochprop.propagation:\detokenize{module-stochprop.propagation}}
\end{sphinxtheindex}

\renewcommand{\indexname}{Index}
\printindex
\end{document}